\documentclass[11pt]{article}
\usepackage{setting}
\usepackage{titlesec}   %设置页眉页脚的宏包
% \usepackage{showkeys}
\usepackage{CJKpunct}
\usepackage{amsmath}
\usepackage{titlesec}
\usepackage{subeqnarray}
\usepackage{cases}
\usepackage{mathrsfs}
\usepackage{verbatim}
\usepackage{ulem}
\usepackage{bm} % 粗且斜
\newcommand*{\QEDA}{\hfill\ensuremath{\blacksquare}}  %自定义,实心
\newcommand*{\QEDB}{\hfill\ensuremath{\square}}  %自定义,空心
\newcommand*{\QEDH}{\hfill{\footnotesize\fzfs 证讫}}  %自定义
\newcommand*{\D}{\text{,}}
\titleformat{\section}{}{{\large\zhxbs 第}\,\large\bf\thesection\,{\large\zhxbs 节}}{10pt}{\large\zhxbs}
\titleformat{\subsection}{}{}{10pt}{\zhht}
\newcommand*{\SL}{\sum\limits}
\newcommand*{\rC}{{\rm C}}
\newcommand*{\bN}{{\bf N}}
\newcommand*{\bZ}{{\bf Z}}
\numberwithin{equation}{section}
\begin{document}
\newpagestyle{main}{
    % \sethead{\tiny 算子}{}{\tiny \thepage}     %设置页眉
    % \headrule                                      % 添加页眉的下划线
    \footrule                                       %添加页脚的下划线
}
\pagestyle{main}    %使用该style
\thispagestyle{empty}
\punctstyle{kaiming}
或许是升幂定理的一个思路:\\
\indent 1. 已知$k \gge 1$,证明$a \equiv b \pmod{p^k} \Rightarrow a^p \equiv b^p \pmod{p^{k+1}}$
\begin{ZM}
    考虑因式分解$a^p-b^p = (a-b)(a^{p-1}+a^{p-2}b + \cdots + b^{p-1})$,欲证明$p^{k+1} \mid a^p-b^p$,只需证$p \mid (a^{p-1}+a^{p-2}b + \cdots + b^{p-1})$,注意到$a \equiv b \pmod{p^k}$,因此$(a^{p-1}+a^{p-2}b + \cdots + b^{p-1}) \equiv pa^{p-1} \pmod{p^k}$,这给出$(a^{p-1}+a^{p-2}b + \cdots + b^{p-1}) \equiv 0 \pmod{p}$,从而证明了结论. \QEDB
\end{ZM}
注意到上面的结论可以加强:按照$k$是最大满足,证明$k+1$是最大满足$\cdots$.然后考察$p$的指数$\cdots$.
\end{document}
