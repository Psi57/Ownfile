\documentclass[10pt]{article}
\usepackage{setting}
\usepackage{titlesec}   %设置页眉页脚的宏包
% \usepackage{showkeys}
\usepackage{CJKpunct}
\usepackage{amsmath}
\usepackage{titlesec}
\usepackage{subeqnarray}
\usepackage{cases}
\usepackage{mathrsfs}
\usepackage{verbatim}
\usepackage{ulem}
\usepackage{bm} % 粗且斜
\newcommand*{\QEDA}{\hfill\ensuremath{\blacksquare}}  %自定义,实心
\newcommand*{\QEDB}{\hfill\ensuremath{\square}}  %自定义,空心
\newcommand*{\QEDH}{\hfill{\footnotesize\fzfs 证讫}}  %自定义
\newcommand*{\D}{\text{,}}
\titleformat{\section}{}{{\large\zhxbs 第}\,\large\bf\thesection\,{\large\zhxbs 节}}{10pt}{\large\zhxbs}
\titleformat{\subsection}{}{}{10pt}{\zhht}
\newcommand*{\SL}{\sum\limits}
\newcommand*{\rC}{{\rm C}}
\newcommand*{\bN}{{\bf N}}
\newcommand*{\bZ}{{\bf Z}}
\newcommand*{\que}[1]{\{#1_n\}}
\numberwithin{equation}{section}
\begin{document}
\titleWithDate{函数算子-线性(暂定\&未完成)}
\newpagestyle{main}{
    \sethead{\tiny 算子}{}{\tiny \thepage}     %设置页眉
    \headrule                                      % 添加页眉的下划线
    % \footrule                                       %添加页脚的下划线
}
\pagestyle{main}    %使用该style
\thispagestyle{empty}
\punctstyle{kaiming}


\par 大约在年初,我从一本书上了解了关于差分算子的概念和应用.如今几个月过去,在阅读了一些资料,博客,教材后,我对算子的了解有所加深.因此,现将算子的概念和它的一些应用整理起来,写在本文.
\par 本文主要利用算子对数列、组合恒等式问题进行一些探讨,也是我的一些小心得.

\section{一些概念的引入}
\par 研究数列,首先需要对数列有个良好的定义.在通常情况下,数列被定义为一列数,而数列的下标规定只能是正整数.可是,某些有通项的数列({\zhfs 本文也主要研究数列的通项})可以被自然地延拓为函数,从而下标也可取为$0$甚至是负整数.为了研究的方便,定义数列是由整数映射到一个数域的函数.如此,研究数列的下标变换就更为方便.有时,数列用函数$f(x)$的形式表示.
\par 对于数域$F$,定义$F_{ZF} = \{f \mid f:\bZ \rightarrow F\}$是定义在$\bZ$的函数.,易验证得$F_{ZF}$构成$F$上的线性空间.而本文的“{\zhht 算子}”指的是$F_{ZF}$上的线性变换.本文主要利用一些特别的算子对数列、组合恒等式等进行研究.
\par 就像对线性变换一样,对于算子$A\D B$与$\lambda \in F$,定义
\[
\begin{split}
A+B: f(x) &\mapsto Af(x)+Bf(x) \\
AB: f(x) &\mapsto A(B(f(x)) \\
\lambda A: f(x) &\mapsto \lambda A(f(x)) \\
\end{split}
\]
标量数乘所定义的算子$\lambda I$可省略$I$,直接记为$\lambda$.
\par 现在,给出本文研究的主要算子
\begin{DY}
    \[\begin{split}
            E: f(x) \mapsto f(x+1)\\
            I: f(x) \mapsto f(x) \\
    \end{split}\]
    \indent 同时,定义$\Delta = E - I$,称为{\zhht 差分算子},$E^{k}: f(x) \mapsto f(x+k)$ ,易知它们都是线性变换. \QEDB
\end{DY}
% \par 在本文,数列被看成是由自然数映射到一数域的函数,为了方便,通常约定这个数域为复数域.同时,定义
\begin{DY}
    对算子$A$,集合$S\subseteq F_{ZF}$定义
    \[\begin{split}
        AS = \{As \mid s \in S\}\D SA = \{sA \mid s \in S\}
    \end{split}\]
\end{DY}
\begin{MT}
    $(\text{{\zhht 消去律}})$ 若算子$A$满足$AF_{NF} = F_{NF}$,则对于算子$B\D C$,有 $BA = CA \Leftrightarrow B = C$
\end{MT}
\begin{ZM}
    $B = C \Rightarrow BA = CA$. \\
    \indent $BA = CA \Rightarrow \forall f \in F_{NF}\D BAf = CAf \Rightarrow \forall f \in AF_{NF} = F_{NF}\D B=C \Rightarrow B=C$. \QEDB
\end{ZM}
\begin{MT}\label{exists}
    对于$E$的非零多项式$f(E)$,有$f(E)F_{NF}=F_{NF}$,这等价于所有递推数列$b_ka_{n+k} + \cdots + b_0a_n = g(n)$有解.
\end{MT}
\begin{ZM}
    只需确定$\{a_n\}$的第$1\D2\D\cdots\D k$项,后面的所有项都可以确定.而非正整数项也可以确定.\QEDB
\end{ZM}
\par 在本文,把$\rC_{n}^{k}\, (n \gge 0)$定义为
\begin{DY}
    \[
        \rC_{n}^{k}=\left\{
        \begin{array}{rcl}
            &\cfrac{n!}{k!(n-k)!}  & {0 \lle k \lle n}\\
            &0  & {k > n \text{或} k < 0}\\
        \end{array} \right.
    \]
\end{DY}
这不是我们平常称的组合数,但是组合数的许多性质被继承了下来,如$\rC_{n}^{k} + \rC_{n}^{k+1} = \rC_{n+1}^{k+1}$,二项展开等,以及$\forall k\D n \gge 0$,$\rC_{n}^{k} = \frac{n(n-1)\cdots (n-k+1)}{k!}$,它可以被看成是$n$的$k$次多项式.
\par 将数列的定义扩大,造成的一个结果是数列的前$n$项和的定义改变.将数列下标的取值扩大到$\bZ$后,数列$\{a_n\}$的前$n$项和$\que{S}$作为数列也应该在非正整数上有取值.现在改变$\que{S}$的定义使其满足这点.一个例子是$a_n = n$,则通常意义下$\que{a}$的前$n$项和为$S_{n} = (n)(n+1)/2$,可以看到,无论$n$的取值是正还是负,$S_n - S_{n-1} = a_n$恒成立.仿此,定义
\begin{DY}
    对一个确定数列$\que{a}$,和数列$\que{S}$是指满足$S_n - S_{n-1} = a_n$且$S_0 = a_0$的数列.
\end{DY}
\begin{MT}
    一个确定数列$\que{a}$的和数列唯一存在.
\end{MT}
\begin{ZM}
    先确定$S_0 = a_0$,然后,对于$n > 0$,使用$S_{n} = a_n + S_{n-1}$来唯一确定$S_{n}$,用数学归纳法可以证明,$S_{n}$唯一确定.对于$n \lle 0$,使用$S_{n-1} = S_n - a_n$来唯一确定$S_{n-1}$,用数学归纳法可以证明,$S_{n-1}$唯一确定.从而$\que{S}$唯一存在.\QEDB
\end{ZM}
可以知道,当$n \gge 0$时,$S_n = a_0 + a_1 + \cdots + a_n$即是平常所称的前$n$项和.
\section{多项式与指数的乘积}
\par 对于数域$F$,定义$F[x]$是$F$上可用多项式表示的数列集合(注意:这和普通的多项式环不同!这里的定义更类似函数),定义:$F[x]_{n}$是$F[x]$中次数不超过$n$的多项式的集合,$F[x]_n$是$n$维线性空间.对于$a \in F$,定义:$a^xF[x]_n = \{g(x) = a^xf(x) \mid f(x) \in F[x]_n\}$,即$a^xF[x]_n$的元素都形如$f(x) = a^x(b_Kx^k + b_{k-1}x^{k-1} + \cdots + b_0)$,其中$k \lle n\D b_0 \cdots b_{k} \in F$.它也是$F$-线性空间.\\
\indent 对于$g(x) \in a^xF[x]$,$g(x)$可以被表示为$a^xf(x)$,$f(x) \in F[x]$.由$f(x) = g(x)/a^x$,用多项式恒等定理可知$f(x)$唯一.在本节,对于$g(x) = a^{x}f(x)\D f(x)\in F[x]$,定义$\deg g = \deg f$.
\begin{MT}
    对于$g(x) = a^{x}F[x]\D \deg g = n\ne 0$,有$\deg (E-a)g(x) = n-1$与$\deg (E-b)g(x) = n \D\forall b \ne a$.
\end{MT}
\begin{ZM}
    显然,$E$在线性空间$a^nF[x]$上的限制也是线性变换({\zhfs\footnotesize 因为$a^{x+1}f(x+1) = a^x(af(x+1))$}).只需对$g(x) = a^{x}x^n$证明此点({\zhfs\footnotesize 因为$a^xF[x]$是线性空间}),$(E-b)g(x) = a^{x}(a(x+1)^n - bx^n)$,当$b \ne a$时,$a(x+1)^n - bx^n$的$x^n$项系数为$a-b\ne 0$.于是$\deg g= n$,当$b= a$时,$x^n$项系数为$0$,$x^{n-1}$项系数为$an$,从而$\deg g = n-1$.\\
    \indent 于是,对于任意$g(x)$,$\deg (E-b)g(x) = n \D \forall b \ne a$是显然的.对于$(E-a)g(x)$,记$g(x) = a^{x}(bx^{n}) + g_0(x)$,$g_0(x) \in a^xF[x]$,$\deg g_0(x) < n$({\zhfs\footnotesize 用多项式的带余除法}),有$(E-a)g(x) = (E-a)a^x(bx^{n}) + (E-a)g_0(x)$,其中$\deg (E-a)g_0(x) < n-1$,$\deg (E-a)(a^x(bx^n)) = n-1$从而$\deg(E-a)g = n-1$.\QEDB
\end{ZM}
\begin{MT}
    对$n > 0$,$(E-a) a^xF[x]_n = a^xF[x]_{n-1}$,$(E-b)a^xF[x]_n = a^xF[x]_n$,$\forall b \ne a$.
\end{MT}
\begin{ZM}
    对于$a^xF[x]_n$的一组基$M = \{a^xx^n\D \cdots \D a^xx\D 1\}$,$(E-a)$作用于$M$后它们分别变为$n-1$阶,$n-2$阶,$\cdots$,$1$阶与$0$.可知$(E-a)M$是$a^xF[x]_{n-1}$的一组基.从而$(E-a)a^xF[x]_n = a^xF[x]_{n-1}$.对于$b \ne a$,$(E-b)M$依然是$a^xF[x]_n$的一组基,从而$(E-b)a^xF[x]_n = a^xF[x]_n$.\QEDB
\end{ZM}
未完待续.\\
所需要完成的:\\
1. $\ker f(E)$的维度.(用同构).\\
2. 证明第5节所需的命题.
\section{算子在组合恒等式中的应用} \label{s:1}
首先,关于$k$的函数$\rC_k^m$,有$\Delta^n \rC_k^m = \rC_k^{m-n}$.于是,在含有$\rC_k^m$项的组合式里,将求和等运算用关于$\Delta$的多项式表示出,有时可以化简出答案.
\par 例如,定义算子$S_n = I + E + \cdots + E^n$,有$(S_nf)(x) = \SL_{k=0}^{n}f(x+k)$,现在,我们尝试把$S_n$化为关于$\Delta$的多项式.就像对等比数列一样,有$(E-I)S_n = E^{n+1}-I = (\Delta + I)^{n+1} - I$,得到
\[
    \Delta S_n = \sum_{k=0}^{n+1}{\rC_{n+1}^k \Delta^k}-I = \Delta\sum_{k=0}^{n}{\rC_{n+1}^{k+1}\Delta^{k}}
\]
这里$\Delta$和后面的项可交换,因此$S_n\Delta = \left(\SL_{k=0}^{n}{\rC_{n+1}^{k+1}}\Delta^k\right)\Delta$,有$S_n = \SL_{k=0}^{n}{\rC_{n+1}^{k+1}}\Delta^k$
\par 例如,令$f(k) = \rC_{k}^{m}$,则$(S_nf)(0) = \SL_{k=0}^{n}{\rC_{k}^{m}} = \SL_{k=0}^{n}{\rC_{n+1}^{k+1}\rC_{0}^{m-k}}$,$\rC_{0}^{m-k}$只在$m = k$时取$1$,其他时候取$0$,从而
$$
\rC_{0}^{m} + \rC_{1}^{m} + \cdots + \rC_{n}^{m} = \rC_{n+1}^{m+1}
$$
由引入的“组合数”的性质,我们有$\rC_{m}^{m} + \rC_{m+1}^{m} + \cdots + \rC_{n}^{m} = \rC_{n+1}^{m+1}$,这个式子被称为{\zhht 朱世杰恒等式}.
\par 现在来看另一个例子:
\begin{LT}
    对于$n \gge m$,证明:$\SL_{k=0}^{m}{\rC_{m}^{k}\rC_{n+k}^m} = \SL_{k=0}^{m}{\rC_{m}^k \rC_n^k2^k}$.
\end{LT}
\begin{ZM}
令数列$a_k = \rC_{n+k}^m$,则 $\SL_{k=0}^{m}{\rC_{m}^{k}\rC_{n+k}^m} = \left(\SL_{k=0}^{m}{\rC_{m}^{k}E^k}\right)a_0 = (E+I)^ma_0 = (\Delta+2)^ma_0 = \SL_{k=0}^{m}{\rC_{m}^{k}2^{m-k}\Delta^k}a_0$.由归纳法可得$\Delta^ka_j = \rC_{n+j}^{m-k}$,于是原式等于$\SL_{k=0}^{m}{\rC_{m}^k\rC_{n}^{m-k}2^{m-k}}$,作指标代换$k = m-k$可得原式等于$\SL_{k=0}^{m}\rC_{m}^{k}\rC_{n}^{k}2^k$.\QEDB
\end{ZM}
\begin{LT}
    证明恒等式:$\SL_{k=0}^{n}{(\rC_n^k)^2\rC_{3n+k}^{2n}} = (\rC_{3n}^n)^2$.
\end{LT}
\begin{ZM}
    我们看到,$\rC_{n}^{x} = (E^{-1}+I)\rC_{n-1}^{x}$$($以$x$为自变量$)$,于是有$\rC_{3n+k}^{2n} = (E^{-1}+I)^{k}\rC_{3n}^{2n} = \SL_{j=0}^{k}{\rC_{k}^j\rC_{3n}^{2n-j}}$,代入原式左边得到:
    \[
        \sum_{k=0}^{n}{(\rC_n^k)^2\rC_{3n+k}^{2n}} = \sum_{k=0}^{n}\sum_{j=0}^{k}{\rC_n^k\rC_{n}^k\rC_{k}^{j}\rC_{3n}^{2n-j}} = \sum_{j=0}^{n}\sum_{k=j}^{n}{\rC_{n}^{k}\rC_{n}^{j}\rC_{n-j}^{n-k}\rC_{3n}^{2n-j}} = \sum_{j=0}^{n}{\rC_{n}^{j}\rC_{3n}^{2n-j}\rC_{2n-j}^{n}} = \sum_{j=0}^{n}{\rC_{n}^j\rC_{3n}^{n}\rC_{2n}^{n-j}} = (\rC_{3n}^n)^2
    \]
    \QEDB
\end{ZM}
\section{算子与高阶等差数列}\label{gjdc}
\subsection{高阶等差数列}
对于给定的数列$\{a_n\}$,差分算子$\Delta$对其作用为$\Delta a_n = a_{n+1}-a_{n}$,称$\{\Delta a_n\}$为$\{a_n\}$的一阶差分数列.一般地,定义$\{\Delta^k a_n\}$为数列$\{a_n\}$的$k$阶差分数列.
\par 对于给定数列$\{a_n\}$,若$\{a_n\}$的$p$阶等差数列不是零数列,而其$p+1$阶等差数列为零数列,则称$\{a_n\}$为$p$阶等差数列.
\par 可以知道,$\Delta^p n^p = p!$,而$\Delta^{p+1} n^p = 0$,所以$\{n^p\}$是$p$阶等差数列,由$\Delta$的线性性,知道若$Q(x)$是$p$次多项式,则数列$\{Q(n)\}$是$p$阶等差数列.而下面这一推论告诉我们,任一$p$阶等差数列,都可以用一$p$次多项式表示.
\begin{TL} \label{tl1:1}
    任一$p$阶等差数列都可以用一$p$次多项式表示.
\end{TL}
\begin{ZM}
    设$\{a_n\}$是一$p$阶等差数列,则$\{\Delta a_n\}$是$p-1$阶等差数列且$a_n = a_0 + \SL_{k=0}^{n-1}{(\Delta a)_k} = a_0 + (S_{n-1}\Delta a)_0$,其中算子$S_n$的定义见第$\ref{s:1}$节.展开得$a_n = a_0 + \SL_{k=0}^{n-1}{\rC_{n}^{k+1}\Delta^{k+1}a_0}$,由于$\{a_n\}$是$p$阶等差数列,结合本文引入的“组合数”的性质,我们有
    $$a_n = a_0 + \sum_{k=0}^{\min(p-1,n-1)}{\frac{n(n-1)\cdots (n-k)}{k!}\Delta^{k+1}a_0} =  a_0 + \sum_{k=0}^{p-1}{\frac{n(n-1)\cdots (n-k)}{k!}\Delta^{k+1}a_0}$$
    由于$\{\Delta^{p+1}a_n\}$是零数列,且$\{\Delta^{p}a_n\}$不是零数列,知道$\{\Delta^{p}a_n\}$应该是不为零的常数数列,则$\Delta^{p}a_0 \ne 0$,说明$a_n =a_0 + \SL_{k=0}^{p-1}{\frac{n(n-1)\cdots (n-k)}{k!}\Delta^{k+1}a_0}$是关于$n$的$p$次多项式.\QEDB
\end{ZM}
\begin{TL}
    $p$次多项式$Q(n)$的前$n$项和$\SL_{k=0}^{n}{Q(k)}$可用$n$的$p+1$次多项式表示.
\end{TL}
\begin{ZM}
    记$a_n = \SL_{k=0}^{n}{Q(k)}$,则$\Delta a_n = Q(n+1)$,由于$Q(n)$是$p$次多项式,说明$\{\Delta a_n\}$是$p$阶等差数列,从而$\{a_n\}$是$p+1$阶等差数列,因此$a_n$可用关于$n$的$p+1$次多项式表示.\QEDB
\end{ZM}
% $(${\zhht 注:}任意$p$阶等差数列都可以用$p$次多项式表示这一结论也可用数学归纳法证明,此处略.$)$
% ------------------------- 以前的内容
\begin{comment}
\par 可以知道,$\Delta^p n^p = p!$,而$\Delta^{p+1} n^p = 0$,所以$\{n^p\}$是$p$阶等差数列,由$\Delta$的线性性,知道若$Q(x)$是$p$次多项式,则数列$\{Q(n)\}$是$p$阶等差数列.而我们会在第 \ref{dt} 节看到,$\ker \Delta^{p+1}$的一组基是$\{1\D n \D \cdots \D n^{p}\}$,说明任一$p$阶等差数列都可以用一$p$次多项式表示.
 \par 若$\{a_n\}$是一$p$阶等差数列,则其前$n$项和$\{S_n\}$满足$\Delta S_n = a_{n+1}$,从而$\Delta^{p+1} S_n = \Delta^{p} a_{n+1}$不是零数列而$\Delta^{p+2} S_n = \Delta^{p+1} a_{n+1}$是零数列,说明$\{S_n\}$是$p+1$阶等差数列.这可以直接推出:
\begin{TL} \label{tl1:1}
    $p$次多项式$Q(n)$的前$n$项和$\SL_{k=0}^{n}{Q(k)}$可用$n$的$p+1$次多项式表示.
\end{TL}
然而,可以发现,在第 \ref{dt} 节证明相关定理时用到了推论\ref{tl1:1}.可以发现要防止循环论证,我们只需用另一种方法证明推论\ref{tl1:1}.
\begin{ZM}
    在第 {\rm \ref{s:1}} 节中有$S_n = I + E + \cdots + E^{n} = \SL_{k=0}^{n}{\rC_{n+1}^{k+1}\Delta^k}$,于是$\SL_{k=0}^{n}{Q(k)} = (S_nQ)(0) = \SL_{k=0}^{n}{\rC_{n+1}^{k+1}}\Delta^{k}Q(0)$.
    由于$Q(n)$是$p$次多项式,因此$\forall k \gge p+1\D \Delta^{k}Q(0) = 0$,从而$S_nQ(0) = \SL_{k=0}^{\min(n,p)}{\rC_{n+1}^{k+1}\Delta^{k}Q(0)}$,根据本文引入的“组合数”的性质,它应该等于$\SL_{k=0}^{p}{\frac{(n+1)n\cdots (n-k+1)}{(k+1)!}\Delta^{k}Q(0)}$,可以被看成是关于$n$的$p+1$次多项式.\QEDB
\end{ZM}
$(${\zhht 注:}任意$p$阶等差数列都可以用$p$次多项式表示这一结论也可用数学归纳法证明,此处略.$)$   
\end{comment}
% -------------------------- 以前的内容
\subsection{推广}
\par 仿造上文,对于$\lambda \ne 0$,定义$\Delta_\lambda = E-\lambda$,现在来求满足$\Delta_\lambda^p a_n = 0$的$a_n$通项具有的形式.
\par 第一种求法仿造上文.根据数列知识,若$b_n = \Delta_\lambda a_n$,即$b_n = a_{n+1} - \lambda a_{n}$,则$a_{n} = c\lambda^{n} + \lambda^{n-1}\SL_{k=0}^{n-1}{b_{k}\lambda^{-k}}$,其中$c$为常数.若$\Delta_\lambda^p a_n = 0$,仿造上文,有$a_n = c\lambda^n + \lambda^{n-1}\SL_{k=0}^{n-1}{\Delta_\lambda a_{k}a^{-k}}$.
\par 记算子$S_n^{(\lambda)} = \SL_{k=0}^{n}{E^k\lambda^{-k}}$,则$(E-\lambda)S_n^{(\lambda)} = E^{n+1}\lambda^{-n} - \lambda = (\Delta_\lambda + \lambda)^{n+1}\lambda^{-n} - \lambda = \SL_{k=0}^{n}{\rC_{n+1}^{k+1}\Delta_\lambda^{k+1}\lambda^{-k}}$,说明$S_{n}^{(\lambda)} = \SL_{k=0}^{n}{\rC_{n+1}^{k+1}\Delta_\lambda^{k}\lambda^{-k}}$.代入$a_n$可得
\[
    a_n = c\lambda^{n} + \lambda^{n-1}\sum_{k=0}^{n-1}{\rC_{n}^{k+1}\lambda^{-k}\Delta_\lambda^{k+1}a_0} = \lambda^{n}\left(c+\sum_{k=0}^{p-1}{\rC_{n}^{k+1}\lambda^{-k-1}\Delta_\lambda^{k+1}a_0}\right)
\]
这说明任意一确定的数列$\{a_n\}$满足$\Delta_\lambda^p a_n = 0$都可以记为$a_n=\lambda^nf(n)$,其中$f(n)$是次数不超过$p$的多项式.
\par 第二种求法利用了高阶等差数列的已知结论且更简洁.对于$\Delta_\lambda^p a_n = 0$,记$b_n = \lambda^{-n}a_n$,则$a_n = \lambda^nb_n$,有$\Delta_\lambda a_n = \lambda^{n+1}(b_{n+1}-b_n) = \lambda^{n+1}(\Delta b_n)$,利用归纳法可知$\Delta_\lambda^k a_n= \lambda^{n+k}(\Delta^{k}b_n)$,由$\Delta_\lambda^p a_n = 0$,知$\lambda^{n+p}(\Delta^p b_n) = 0$,说明$b_n$是可以由次数不超过$p$的多项式表示,从而知道$a_n = \lambda^{n}f(n)$,$f(n)$是次数不超过$p$的多项式.
\par 同时,任意一$a_n = \lambda^nf(n)$,都满足$\Delta_\lambda^k a_n = \lambda^{n+k}(\Delta^k f(n))$,说明若$f(n)$是次数不超过$p$的多项式,则$\Delta_\lambda^{p} a_n = 0$.
\par 这一小节可以直接应用在第 \ref{dt} 节中,而且第 \ref{dt} 节还提供了一种数学归纳法的证明.
\section{算子与常线性递推数列} \label{dt}
\par 任一关于数列$\{a_n\}$的递推公式$b_ka_{n+k}+b_{k-1}+a_{n+k-1}+\cdots+b_{0}a_{n}=g(n)$,$($其中$g(n)$是一给定函数,$b_0\D \cdots \D b_k$是常数,$b_k \ne 0$$)$,都可以看成是$\left(b_kE^{k} + \cdots + b_0I\right)a_n = g(n)$.我们先探讨最简单的情况:
\begin{YL} \label{yl1}
    满足递推式$a_{n+1}-ba_{n} = g(n)$的通解为$a_n = cb^n+\SL_{k=0}^{n-1}{g(k)b^{n-1-k}}$,其中$c$为常数.
\end{YL}
这个结论的证明可以由简单的递推数列的知识得到,证明略.
\subsection{常线性齐次递推数列}
\par 首先,我们研究$(E-a)^ka_n = 0$的通解,即求出$\ker (E-a)^k$,其中$a \ne 0$.其实第 \ref{gjdc} 节已经给出了答案,这里再给一个归纳法的证明.第一步,由引理$\ref{yl1}$得
$$
(E-a)a_n = 0 \Leftrightarrow a_n = C_1a^n \text{,其中$C_1$为常数.}
$$
那么,$(E-a)^2a_{n} = 0$等价于$(E-a)\left((E-a)a_n\right) = 0$,可以推出$(E-a)a_n = C_1a^n$,再次利用引理$\ref{yl1}$得$a_n = C_2a^n+C_1\SL_{k=0}^{n-1}{a^ka^{n-1-k}} = C_2a^n + C_1(n-1)a^{n-1} = (C_2-C_1a^{-1})a^{n} + a^{-1}C_1na^n$,注意到$C_1\D C_2$是常数,调整$C_1\D C_2$可得$a_n = C_2a^{n} + C_1na^{n}$.像这样,反复利用引理$\ref{yl1}$,可以得到
\begin{TL}
    $\ker (E-a)^k$的一组基为$\{a^n \D na^n \D \cdots n^{k-1}a^n\}$,其中$a\ne 0$.
\end{TL}
\begin{ZM}
    首先:若$g(n)$是关于$n$的$k$次多项式,则$\SL_{k=0}^{n}{g(k)}$是关于$n$的$k+1$次多项式.\\
    \indent 利用数学归纳法.当$k=1$时,已验证得命题成立.\\
    \indent 若$k=m-1$时命题已经成立,则$k=m$时:
    \[\begin{split}
            (E-a)^ka_n = 0 &\Leftrightarrow (E-a)a_n = C_0a^n+C_1na^n+\cdots+C_{k-2}a^{k-2}a^n \\
            &\Leftrightarrow a_n = Ca^n + C_0f_1(n)a^{n-1} + C_1f_2(n)a^{n-1}+ \cdots + C_{k-2}f_{k-1}(n)a^{n-1}
    \end{split}\]
    其中$C\D \cdots \D C_{k-2}$是常数,$f_{k}(n) = \SL_{i=0}^{n-1}{i^{k-1}}$是关于$n$的$k$次多项式.\\
    \indent 可见$\ker (E-a)^{k}$的一组基为$\{a^n \D na^n \D \cdots n^{k-1}a^n\}$.由数学归纳法原理知推论成立.\QEDB
\end{ZM}
现在,给出一个引理
\begin{YL}\label{yl2}
    若$f_1(x)\D f_2(x)\D \cdots\D f_n(x)$是数域$F$上两两互素的多项式,$\mathscr{A}$是定义在数域$F$上的线性空间$V$的线性变换,则$\ker f_1({\mathscr{A}})f_2({\mathscr{A}})\cdots f_n({\mathscr{A}}) = \ker f_1({\mathscr{A}}) \oplus \ker f_2({\mathscr{A}}) \oplus \cdots \oplus \ker f_n({\mathscr{A}})$.
\end{YL}
\begin{ZM}
    先证明$n=2$的情况.首先,证明$\ker f_1({\mathscr{A}}) \oplus \ker f_2({\mathscr{A}})$ 是直和.
    \par 若${\alpha} \in \ker f_1({\mathscr{A}})\D \beta \in \ker f_2({\mathscr{A}})$有$\alpha + \beta = {\bf{0}}$,则在等式两边用$f_1({\mathscr{A}})$作用得$f_1({\mathscr{A}})(\beta) = \bf{0}$,所以$\beta \in \ker f_1({\mathscr{A}}) \cap \ker f_2({\mathscr{A}})$. \\
    \indent 由于$f_1\D f_2$互素,存在多项式$v(x)\D u(x)$,使得$v(x)f_1(x)+u(x)f_2(x)=1$,于是$v({\mathscr{A}})f_1({\mathscr{A}})(\beta)+u({\mathscr{A}})f_2({\mathscr{A}})(\beta)=\beta$,从而$\beta = \bf{0}$.同理$\alpha = \bf{0}$,这说明$\ker f_1({\mathscr{A}}) \oplus \ker f_2({\mathscr{A}})$是直和. \\
    \indent 再证$\ker f_1({\mathscr{A}}) \oplus \ker f_2({\mathscr{A}}) = \ker f_1({\mathscr{A}})f_2({\mathscr{A}})$.由于$f_1\D f_2$互素,存在多项式$v(x)\D u(x)$,使得$v(x)f_1(x)+u(x)f_2(x)=1$,于是对于$\alpha \in \ker f_1({\mathscr{A}})f_2({\mathscr{A}})$,$\alpha = v({\mathscr{A}})f_1({\mathscr{A}})(\alpha)+u({\mathscr{A}})f_2({\mathscr{A}})(\alpha)$,其中$v({\mathscr{A}})f_1({\mathscr{A}})(\alpha) \in \ker f_2({\mathscr{A}})\D u({\mathscr{A}})f_2({\mathscr{A}})(\alpha) \in \ker f_1({\mathscr{A}})$,说明$\ker f_1({\mathscr{A}})f_2({\mathscr{A}})$上任意向量都可以表示为$\ker f_1({\mathscr{A}})$,$\ker f_2({\mathscr{A}})$上向量的和.可见$\ker f_1({\mathscr{A}})f_2({\mathscr{A}}) = \ker f_1({\mathscr{A}}) \oplus \ker f_2({\mathscr{A}})$. \\
    \indent 于是,由于$f_1(x)\D \cdots f_n(x)$两两互素,所以
    \[\begin{split}
        \ker f_1({\mathscr{A}})f_2({\mathscr{A}})\cdots f_n({\mathscr{A}}) &= \ker f_1({\mathscr{A}}) \oplus \ker f_2({\mathscr{A}})\cdots f_n({\mathscr{A}})\\ 
        &= \ker f_1({\mathscr{A}}) \oplus \ker f_2({\mathscr{A}}) \oplus \ker f_3({\mathscr{A}})\cdots f_n({\mathscr{A}}) \\ 
        &= \cdots\\
        &= \ker f_1({\mathscr{A}}) \oplus \ker f_2({\mathscr{A}}) \oplus \cdots \oplus \ker f_n({\mathscr{A}})
    \end{split}\]. \QEDB
\end{ZM}
\par 有了这个引理后,对于$t$个互不相同的数$a_1\D \cdots a_t$,我们可以直接推出
\[
    \ker(E-a_1)^{k_1}(E-a_2)^{k_2}\cdots (E-a_t)^{k_t} = \ker(E-a_1)^{k_1} \oplus \cdots \oplus \ker(E-a_t)^{k_t}
\]
从而有
\begin{TL}
    $\ker(E-a_1)^{k_1}\cdots (E-a_t)^{k_t} \text{的一组基为} \{a_1^n\D na_1^n \D \cdots n^{k_1-1}a_1^n \D \cdots \D a_{t}^n \D \cdots \D n^{k_t-1}a_{t}^n\}$,其中$a_1\D \cdots a_t$互不相同.\QEDB
\end{TL}

下面阐述求常线性齐次递推数列通解的算法:\\
\indent 对于$b_ka_{n+k}+\cdots+b_0a_n = 0\,(b_k \ne 0)$,解出$b_kx^k+\cdots+b_0 = 0$的所有复根$x_1\D x_2 \D \cdots x_t$,记它们的重数分别为$k_1\D k_2 \D \cdots k_t$,则$\{a_n\}$的通解为$a_n = C_{1,1}x_1^n + C_{1,2}nx_1^n + \cdots + C_{1,k_1}n^{k_1-1}x_1^n + \cdots + C_{t,1}x_t^n + \cdots + C_{t,k_t}n^{k_t-1}x_t^n$,其中$C_{i,j}$皆为常数$(\text{待定系数})$.
\par 这种方法被称为{\zhht 特征根法},其中$b_kx^k+\cdots+b_0 = 0$称为递推式$b_ka_{n+k}+\cdots+b_0a_n = 0$的{\zhht 特征方程}.
\subsection{常线性非齐次递推数列}
就像解非齐次线性方程组一样,若知道递推$(E-a_1)^{k_1}(E-a_2)^{k_2}\cdots (E-a_t)^{k_t}a_n = g(n)$的一个解$b_n$,我们可以推出所有解都有形式$a_n = b_n + c_n$,其中$c_n$是$(E-a_1)^{k_1}(E-a_2)^{k_2}\cdots (E-a_t)^{k_t}c_n = 0$的任意解.而且只要重复利用引理\ref{yl1},我们可以求出一个特解.
\par 但大多时候,求和并不是那么方便.方便的方法之一是待定系数法.若$g(n)$满足$(E-b_1)^{m_1}\cdots (E-b_l)^{m_l}g(n) = 0$,那么在递推式两边用$(E-b_1)^{m_1}\cdots (E-b_l)^{m_l}$作用可以得到$(E-b_1)^{m_1}\cdots (E-b_l)^{m_l}(E-a_1)^{k_1}(E-a_2)^{k_2}\cdots (E-a_t)^{k_t}a_n = 0$,求得$a_n$的通解形式后再用$(E-a_1)^{k_1}(E-a_2)^{k_2}\cdots (E-a_t)^{k_t}$作用于$a_n$使得$a_n$的待定系数满足$(E-a_1)^{k_1}(E-a_2)^{k_2}\cdots (E-a_t)^{k_t}a_n = g(n)$.
\begin{LT}
    求$a_n = n^22^n$的前$n$项和$S_n$.
\end{LT}
\begin{JT}
    $S_n$满足$(E-I)S_n = (n+1)^22^{n+1}$,两边用$(E-2)^3$作用得$(E-2)^3(E-1)S_n = 0$,于是$S_n$有形式$C_1+C_22^n+C_3n2^n+C_4n^22^n$,$a_n = S_n - S_{n-1} = C_4n^22^{n-1}+n2^{n-1}(C_3+2C_4)+2^{n-1}(C_2-C_4+C_3)$,和$a_n = n^22^n$比较得
    \[
        \left\{
        \begin{aligned}
            &C_4 = 2 \\
            &C_3+2C_4 = 0 \\
            &C_2-C_4+C_3 =  0
        \end{aligned}
        \right.
    \]
    解之,得$S_n = C_1+6\times 2^n-4n2^n+2n^22^n$,将$S_1 = a_1$代入得$S_n = -6+6\times 2^n-4n2^n+2n^22^n$.\QEDB
\end{JT}
$($闲话:Q:为什么$C_4n^22^{n-1}+n2^{n-1}(C_3+2C_4)+2^{n-1}(C_2-C_4+C_3)$和$n^22^n$相等则系数一定对应相等呢?\\
A:因为$2^n\D n2^n \D n^22^n${\zhht 线性无关}!$)$
\par 另一种常见的方法是配凑,即对于$f(E)a_n = g(n)$,求出一个$b_n$使得递推式化为$f(E)(a_n-b_n) = 0$.$b_n$等价于$f(E)a_n = g(n)$的一个解.有时若$g(n)$是多项式函数,我们可能会尝试用多项式数列$b_n$来对原递推式进行配凑,并且有时$b_n$的次数和$g(n)$一样,有时$b_n$的次数却比$g(n)$高.下面的推导来验证多项式配凑的可行性.并且给出$b_n$次数的具体取值.
\par 在这里,记$F_k[x]$表示数域$F$上以$x$为字母的所有次数不超过$k$的多项式的集合.
\begin{YL}
    若$a \ne 1$,则$(E-a)F_k[x] = F_k[x]$,并且$(E-I)F_{k+1}[x] = F_{k}[x]$.
\end{YL}
\begin{ZM}
    若$a \ne 1$,对任意多项式$f(x)$,记其首项系数为$b \ne 0$,则$f(x+1)-af(x)$的首项系数为$b-ab$,由$a \ne 1$知$b-ab \ne 0$,从而$\deg (E-a)f(x) = \deg f(x)$.可见$(E-a)$在线性空间$F_k[x]$上的限制为线性变换.欲证$(E-a)F_k[x] = F_k[x]$,只需证明$\ker (E-a)|_{F_k[x]}$只有零.由上文可得,$f(x) \in F_k[x]$的最小多项式有形式$(\lambda - 1)^m$,而$(\lambda - 1)$与$(\lambda - a)$互素,这说明$\ker (E-a)|_{F_k[x]}$只含零,从而证明了$(E-a)F_k[x] = F_k[x]$.
    \par 对于$(E-I)$,由第 $\ref{gjdc}$ 节的知识有$\forall f(x) \in F_{k+1}[x]\D (E-I)f(x) \in F_{k}[x]$,说明$(E-I)F_{k+1}[x] \subseteq F_{k}[x]$.另一方面,$\forall f(x) \in F_k[x]$,取$g(x)=\SL_{k=0}^{n-1}{f(k)}$便有$(E-I)g(x) = f(x)$,其中$g(x) \in F_{k+1}[x]$,所以$(E-I)F_{k+1}[x] \supseteq F_{k}[x]$,因此$(E-I)F_{k+1}[x] = F_k[x]$.\QEDB
\end{ZM}
\begin{TL}
    对于$g(x) \in F_m[x]$,存在多项式$f(x) \in F_{m+k}[x]$使得$(E-a_1)^{k_1} \cdots (E-a_t)^{k_t}(E-I)^kf(x) = g(x)$.
\end{TL}
\begin{ZM}
    由$(E-a_1)^{k_1}\cdots (E-a_t)^{k_t}(E-I)^k F_{m+k}[x] =(E-a_1)^{k_1}\cdots (E-a_t)^{k_t}F_m[x] = F_m[x]$知推论成立.\QEDB
\end{ZM}
上述推论证明了对于递推式$f(E)a_n = g(n)$,且$g(n)$是关于$n$的多项式,若$f(E)$中含$(E-I)$因子的个数为$k$,$\deg g(n) = m$,则可以用$m+k$次多项式进行配凑.
\par 当然,配凑还可以为差比形$(a^nf(n))$,它们具体的性质和多项式配凑非常相似,此处略过.
\section{常线性递推数列组}
在上一节我们研究了常线性递推数列的解法,在本节,我们研究常线性递推数列组.这里用一个例题作为开头.
\begin{LT}
    求满足下列式子的$\{a_n\}$,$\{b_n\}$的通解.
    \[
        \left\{
            \begin{aligned}
                &a_{n+2} = -3a_{n+1}+4a_n-6b_{n+1}-12b_n \\
                &b_{n+2} = -b_{n+1}+2b_n+a_{n+1}-a_n
            \end{aligned}
        \right.
    \]
\end{LT}
\begin{JT}
这是$a_n\D b_n$递推式的杂糅交错形成的方程组,稍微移项得
\[
        \left\{
            \begin{aligned}
                &a_{n+2}+3a_{n+1}-4a_n + 6b_{n+1} + 12b_n = 0\\
                &b_{n+2} + b_{n+1}-2b_n - a_{n+1} + a_n= 0
            \end{aligned}
        \right.
\]
或者说
\[
        \left\{
            \begin{aligned}
                &(E^2+3E-4)a_n + 6(E+2)b_n = 0\\
                &-(E-1)a_n + (E^2+E-2)b_n = 0
            \end{aligned}
        \right.
\]
这时,如果我们把它写成矩阵形式:
\[
    \begin{pmatrix} (E-1)(E+4) & 6(E+2) \\ -(E-1) & (E-1)(E+2) \end{pmatrix} \begin{pmatrix}
    a_n \\ b_n\end{pmatrix} = {\bm{O}}
\]
将$\lambda$矩阵$\begin{pmatrix} (\lambda-1)(\lambda+4) & 6(\lambda+2) \\ -(\lambda-1) & (\lambda-1)(\lambda+2) \end{pmatrix}$记为$\bm {A}(\lambda)$,对$\bm{A}(\lambda)$进行初等变换使其成为对角方阵:
\[ \begin{split}
        & \begin{pmatrix} (\lambda-1)(\lambda+4) & 6(\lambda+2) \\ -(\lambda-1) & (\lambda-1)(\lambda+2) \end{pmatrix} \xrightarrow[]{(1,2)} \begin{pmatrix} -(\lambda-1) & (\lambda-1)(\lambda+2) \\ (\lambda-1)(\lambda+4) & 6(\lambda+2) \end{pmatrix} \\ &\xrightarrow[]{(\lambda+4)(1) + (2)} \begin{pmatrix} -(\lambda-1) & (\lambda-1)(\lambda+2) \\ 0 & (\lambda+1)(\lambda+2)^2 \end{pmatrix} \xrightarrow[(\lambda+2)(1)+(2)]{} \begin{pmatrix} -(\lambda-1) & 0 \\ 0 & (\lambda+1)(\lambda+2)^2 \end{pmatrix}
\end{split}\]
记${\bm \Lambda}(\lambda) = \begin{pmatrix} -(\lambda-1) & 0 \\ 0 & (\lambda+1)(\lambda+2)^2 \end{pmatrix}$,${\bm Q}(\lambda)  =  \begin{pmatrix} 1 & \lambda+2 \\ 0 & 1 \end{pmatrix}$与${\bm P}(\lambda)  = \begin{pmatrix} 0 & 1 \\ 1 & \lambda+4\end{pmatrix}$,则上述变换告诉我们$\bm{P}(\lambda) \bm{A}(\lambda) \bm{Q}(\lambda) = \bm{\Lambda}(\lambda) $,因此原方程就是$\bm{P}^{-1}(E) \bm{\Lambda}(E) \bm{Q}^{-1}(E) \bm{X} = \bm{O}$$\Leftrightarrow \bm{\Lambda}(E) \bm{Q}^{-1}(E) \bm{X} = \bm{O}$$\Leftrightarrow \bm{Q}^{-1}(E) \bm{X} = \begin{pmatrix}C_1 \\ C_2(-1)^n + C_3(-2)^n + C_4(-2)^nn\end{pmatrix}$\\$ \Leftrightarrow \bm{X} = \begin{pmatrix}C_1 + C_2(-1)^n + C_4(-2)^{n+1} \\ C_2(-1)^n + C_3(-2)^n + C_4(-2)^nn \end{pmatrix}$
这说明原递推的通解为
\[
        \left\{
            \begin{aligned}
                &a_n = C_1 + C_2(-1)^n + C_4(-2)^{n+1} \\
                &b_n = C_2(-1)^n + C_3(-2)^n + C_4(-2)^nn
            \end{aligned}
        \right.
\]
\QEDB
\end{JT}
一般地,对于常系数线性递推
\[
        \left\{
            \begin{aligned}
                &a^{(1)}_{n+k} = F_1(a^{(1)}_n\D\cdots a^{(1)}_{n+k-1}\D\cdots\D a^{(m)}_{n+k-1}\D n)\\
                &a^{(2)}_{n+k} = F_2(a^{(1)}_n\D\cdots a^{(1)}_{n+k-1}\D\cdots\D a^{(m)}_{n+k-1}\D n)\\
                &\cdots \\
                &a^{(m)}_{n+k} = F_m(a^{(1)}_n\D\cdots a^{(1)}_{n+k-1}\D\cdots\D a^{(m)}_{n+k-1}\D n)
            \end{aligned}
        \right.
\]
求$\{a^{(i)}_n\}$通解,我们可以把它写成矩阵形式$\bm{A}(E)\bm{X} = \bm{Y}$,作初等变换使$\bm{A}(\lambda) = \bm{P}^{-1}(\lambda)\bm{\Lambda}(\lambda)\bm{Q}^{-1}(\lambda)$,将$\bm{A}(\lambda)$相抵于对角阵$\bm{\Lambda}{(\lambda)}$,则原递推的通解可以表示为$\bm{X} \in \bm{Q}(E)\bm{\Lambda}^{-1}(E)\bm{P}(E)\bm{Y}$,其中$\bm{\Lambda}^{-1}(E)\bm{S} = \{\bm{K} \mid \bm{\Lambda}(E)\bm{K} = \bm{S}\}$.由于$\bm{\Lambda}(E)$是对角阵,解$\bm{\Lambda}^{-1}(E)\bm{P}(E)\bm{Y}$其实是解$m$个常线性(非)齐次递推数列.注意,若$\bm{Y} \ne \bm{O}$,则原递推式可能无解.
\par 现在,一个问题是,$\bm{A}(E)\bm{X} = \bm{Y}$的解空间的维数是多少?按照上文,将$\bm{A}(\lambda)$相抵成对角阵$\bm{A}(\lambda) = \bm{P}^{-1}(\lambda)\bm{\Lambda}(\lambda)\bm{Q}^{-1}(\lambda)$,于是原式等价于$\bm{\Lambda}(E)\bm{Q}^{-1}(E)\bm{X} = \bm{P}(E)\bm{Y}$,知\\
\indent{\zhht 情况一} 若存在正整数$i$使得$\bm{\Lambda}(\lambda)$的第$i$行$i$列元素为$0$但是$\bm{Y}$的第$i$行不为零,则原式无解. \\
\indent{\zhht 情况二} 若不满足情况一,对$\bm{\Lambda}(\lambda)$每个非零多项式$f_{i}(\lambda)$,记$\deg f_i(\lambda) = m_i$,$f_i(\lambda) = \lambda^{t_i}g_i(\lambda)$并且$\lambda \nmid g_i(\lambda)$,则原式解空间的维度为所有$m_i-t_i$的和.
\begin{LT}
    求满足下列式子的通解:
    \[
        \left\{
            \begin{aligned}
                &a_{n+2} = a_{n+1}+2a_n+b_{n+1}+b_n \\
                &b_{n+2} = -2b_{n+1}-b_n+a_{n+1}-2a_n
            \end{aligned}
        \right.
    \]
\end{LT}
\begin{JT}
    记$\bm{A}(\lambda) = \begin{pmatrix}(\lambda^2-\lambda-2) & -(\lambda+1) \\-(\lambda -2) & (\lambda+1)^2 \end{pmatrix}$,则原式即求$\bm{A}(E)\bm{X} = \bm{O}$的解.对$\bm{A}(\lambda)$作初等变换得\\$\bm{A}(\lambda) = 
    \begin{pmatrix}
        1 & 0 \\
        \lambda+1 & 1
    \end{pmatrix}^{-1}
    \begin{pmatrix}
        -(\lambda+1) & 0 \\
        0 & \lambda(\lambda+2)(\lambda-2)
    \end{pmatrix}
    \begin{pmatrix}
        0 & 1 \\
        1 & \lambda-2
    \end{pmatrix}^{-1}
    $
    ,从而原式的通解为
    \[
        \left\{
            \begin{aligned}
                &a_{n} = C_2(-2)^n + C_32^n \\
                &b_{n} = C_1(-1)^n + 2C_2(-2)^{n+1}
            \end{aligned}
        \right.
    \]
\end{JT}
\begin{LT}
    解方程:
    \[
        \begin{pmatrix}
            (E-2)^2 & E^2-1 \\
            4E-11 & 4E+5
        \end{pmatrix}\bm{X} = 
        \begin{pmatrix}
            n \\ n^2
        \end{pmatrix}
    \]
\end{LT}
\begin{JT}
    由于$
        \begin{pmatrix}
            (E-2)^2 & E^2-1 \\
            4E-11 & 4E+5
        \end{pmatrix}^{-1} = \cfrac{1}{9}
        \begin{pmatrix}
            4E+5 & 1-E^2 \\
            11-4E & (E-2)^2
        \end{pmatrix}
    $,在式子两边左乘它可得
    \[
        \bm{X} = \cfrac{1}{9}\begin{pmatrix}
            5n \\ n^2+3n-4
        \end{pmatrix}
    \]
\end{JT}
\par 在上面的步骤中,矩阵的作用实际上是递推式的下标变换,加减消元(对应左乘)和换元(对应右乘).例如第一题将矩阵的作用翻译过来,则是
\begin{JT}
    \begin{subequations}
       \begin{numcases}{}
            a_{n+2}+3a_{n+1}-4a_n + 6b_{n+1} + 12b_n = 0 \label{5:1}\\
            b_{n+2} + b_{n+1}-2b_n - a_{n+1} + a_n= 0 \label{5:2}
        \end{numcases}
    \end{subequations}
    对 \eqref{5:2} 作下标变换$n = n+1$得到
    \begin{equation}
        b_{n+3}+b_{n+2}-2b_{n+1}-a_{n+2}+a_{n+1} = 0 \label{5:3}
    \end{equation}
计算 $\eqref{5:3} + 4 \eqref{5:2} + \eqref{5:1}$得到
\begin{equation}
    b_{n+3}+5b_{n+2}+8b_{n+1}+4b_n = 0 \label{5:4}
\end{equation}
将$b_{n+1} + 2b_{n}-a_{n}$看成整体,由\eqref{5:2}有
\begin{equation}
    (b_{n+2}+2b_{n+1}-a_{n+1})-(b_{n+1}+2b_{n}-a_{n}) = 0 \label{5:5}
\end{equation}
由\eqref{5:4}式知$b_n = C_2(-1)^n + C_3(-2)^n + C_4(-2)^nn$,从而$b_{n+1}+2b_{n} = C_2(-1)^n + C_4(-2)^{n+1}$,\eqref{5:5}告诉我们$b_{n+1}+2b_{n}-a_{n}$是常数数列,于是$a_{n} = C_1 + C_2(-1)^n + C_4(-2)^{n+1}$,验证可得原递推的通解为
\[
        \left\{
            \begin{aligned}
                &a_n = C_1 + C_2(-1)^n + C_4(-2)^{n+1} \\
                &b_n = C_2(-1)^n + C_3(-2)^n + C_4(-2)^nn
            \end{aligned}
        \right.
\]
其中$C_1\D C_2\D C_3\D C_4$为常数.\QEDB
\end{JT}
\par 《九章算数》曾说:“方程以御错糅正负”,从这一小节的例子看到,无论是定义在数域上的方程组问题,还是定义在算子多项式上的“递推数列组”,都可以用矩阵解决,这么看来矩阵的方法是解决线性问题的一条康庄大道.
\section{差分与微分}
\section{一些变系数线性递推数列}
\section{更多的例子}
\subsection{片段和}
对于一个数列$\{a_n\}$,记其前$n$项和为$\{S_n\}$,则$\{a_n\}$从$n+1$开始长度为$m$的片段和可以表示为$S_{n+m}-S_{n}$,或者说$(E^m-I)S_n$,记为$\{T_n\}$,则$T_n = (E^m-I)S_{n} = \frac{E^m-I}{E-I}a_{n+1}$.若$a_n$关于$E$有最小多项式$d_a(\lambda)$,则$T_n$的最小多项式是$\frac{d_a(\lambda)}{\left(d_a(\lambda),\frac{\lambda^m-1}{\lambda-1}\right)}$,这就可以确定$T_n$的形式.举两个具体的例子:等差数列的片段和依然是等差数列,$a_n = 2^nn$的片段和有形式$2^n(An+B)$.
\begin{LT}
    已知数列$($下标从$1$开始$)$$c_n = a_nb_n$,其中$\{a_n\}$是等差数列,$\{b_n\}$是等比数列且公差、公比为实数.记$\{c_n\}$的前$n$项和为$\{S_n\}$,$m$为一正偶数,若$S_m = 11$,$S_{2m} = 7$,$S_{3m} = -201$,求$S_{4m}$.
\end{LT}
\begin{JT}
    不妨记$S_0 = 0$,记$b_n$的公比为$q$.可以看到,$S_{(n+1)m}-S_{nm} = c_{nm+1}+\cdots+c_{(n+1)m}$,是$c_n$以$nm+1$为起点,长度为$m$的片段和.记$c_{n+1} + \cdots + c_{n+m} = T_n$,则$T_n$有零化多项式$(\lambda-q)^2$,$T_n$具有形式$q^n(An+B)$,则$S_{(n+1)m} - S_{nm} = T_{nm} = q^{mn}(Anm+B)$.记$T_{nm} = c'_n\D p = q^m$.则$(E-p)^2c'_n = 0$,即
    $$c'_{n+2} - 2p c'_{n+1} + p^2c'_n = 0$$
    将$c'_{0} = 11\D c'_1 = -4\D c'_2 = -208$代入得$11p^2+8p-208 = 0$,解得$p_1 = 4\D p_2 = -\frac{52}{11}$,由于$p = q^m \gge 0$,$p = 4$.因此$c'_3 = 2pc'_2-p^2c'_1 = 8c'_2-16c'_1=-1600$,从而$S_{4m} = -1600+S_{3m} = -1801$. \QEDB
\end{JT}
\section{尚未完成,仍需补充的内容}
\noindent 阿贝尔变换\\
组合恒等式、高阶等差数列的例子\\
\sout{常线性非齐次递推数列的多项式配凑的可行性}\\
\sout{高阶等差数列对于幂函数乘以多项式时的推广,用它来证明特征根法}\\
幂和问题\\
变系数线性递推数列中算子的非交换性\\
未完成的章节\\
收藏夹$CCC$的内容\\
\sout{高阶等差数列推论说不定能换个更简单的解法?}可以\\
\sout{常线性递推数列组中初等变换的直观意义}\\
\sout{简单数列的片段和}\\
简单函数递推\\
更合理的写作顺序:从定义出发,证明性质与给出应用.\\
组合式的最后一个例题,写好一点.\\
这里提一下:写作主题和顺序.这里,第2、3、4节在各自主题外都在为第5节做相同的铺垫.整理一下.\\
\today\LaTeX
\end{document}
